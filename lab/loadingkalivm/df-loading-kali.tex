\documentclass[a4paper,11pt]{article}

\usepackage{graphicx}
\usepackage[printwatermark]{xwatermark}
\usepackage[margin=1in]{geometry}
\usepackage[doublespacing]{setspace}
\usepackage{color} % needed if you use xfig graphics at all
\usepackage{xcolor}
\usepackage{url}
\usepackage{hyperref}
\usepackage{csquotes}
\usepackage{listings}
\usepackage{courier}
\usepackage{subfigure}
\usepackage{mdwlist}
\usepackage{tabu}
\usepackage{array,etoolbox}
\usepackage{lscape}
\usepackage{rotating}
\usepackage[toc,page]{appendix}
\usepackage{pdfpages}
\usepackage{pdflscape}	%allows landscape orientation for floats
\usepackage{afterpage}
\usepackage{ragged2e}
\usepackage{textcomp}
\usepackage{mdwlist}
\usepackage{amsmath}	% allows \text{} in math mode $$ or \[ \]
\usepackage{amsthm}		%used for defining theorems
\usepackage{longtable} % for 'longtable' environment




\author{Fraser Brown}
%\institute{Heriot-Watt University}
\date{\today}

\graphicspath{{../figures/}}

\begin{document}

%Title
   \begin{center}
      \Large\textbf{Loading Kali Linux into VirtualBox}\\
      \large\textit{File System Forensics Lab}\\
      \large\textit{Fraser Brown - Software Engineering MEng}
   \end{center}
\section{Lab Environment}
In order to create a Kali Linux VM, \textbf{copy the file} form my \texttt{Public} directory into \texttt{your} \texttt{/scratch/} drive as shown below:\\
\begin{tabular}{|c|}
	\hline
		\texttt{{\#} cp /home/cs4/fmb30/Public/Kali-Linux-2018.1-vbox-amd64.ova /scratch/} \\
	\hline
\end{tabular}

\begin{enumerate}
	\item \textbf{Before Loading the \texttt{Kali} VM}, open \texttt{virtualbox} and change the ``Default Machine Folder'' to \texttt{/scratch/} (in short, you can change this option after selecting\\ \verb|File --> Preferences --> General|
	\item \textbf{Then, load the \texttt{Kali} VM} into \texttt{virtualbox} as follows:
	\begin{enumerate}
		\item Choose \verb|File --> Import Appliance|
		\item For \texttt{Appliance to Import}, find \texttt{/scratch/Kali-Linux-2018.1-vbox-amd64.ova} and click Next.
		\item For \texttt{Appliance Settings}, select \texttt{Reinitialize the Mac Address of all network cards} and then choose Import.
	\end{enumerate}
	\item \textbf{After loading the VM, right-click} on the \texttt{Kali} VM in \texttt{virtualbox} to select \texttt{Settings} and do the following: 
	\begin{enumerate}
		\item Under \texttt{General-->Basic}, set the \texttt{Name} of the VM to whatever you wish.
		\item Under \texttt{System-->Motherboard}, change the \texttt{Memory Size} from \texttt{2048MB} to \texttt{1024MB}.
	\end{enumerate}
\end{enumerate}

You should now have an installed \texttt{Kali} VM installed in \texttt{virtualbox}.


\end{document}